\documentclass[12pt]{article}
\usepackage[utf8]{inputenc}
\usepackage[english]{babel}
\usepackage[letterpaper, portrait, margin=1in]{geometry}
\usepackage{amsmath}
\numberwithin{equation}{section}
\usepackage{amssymb}
\usepackage{graphicx}
\usepackage{parskip}
\usepackage{xcolor}
\usepackage{physics}
\usepackage{empheq}
\usepackage{cancel}
\usepackage{hyperref}
\hypersetup{colorlinks = true, urlcolor = blue, linkcolor = red, citecolor = red}
\usepackage{enumerate}
\usepackage{tikz}
\usepackage{float}
\usepackage{tcolorbox}
\usepackage{booktabs}
\usepackage[bottom]{footmisc}

\usepackage{xcolor}
\usepackage{fancyhdr}
\pagestyle{fancy}
\fancyhf{}
\fancyfoot[C]{\color{lightgray}}
\fancyfoot[L]{\color{lightgray} \today}
\fancyfoot[R]{Page \thepage}
\renewcommand{\headrulewidth}{0pt}
\renewcommand{\footrulewidth}{0pt}

\begin{document}
	\begin{center}
			\textbf{\Large{ULAB Physics \& Astronomy, Installation Guide}}
	\end{center}

	\tableofcontents

	\section{Preface}
	
	This guide will serve as the instructions for installing various software and packages that we will be using throughout the semester. The specific installation process may vary based on the operating system of your computer. If you encounter any issues, reach out to the ULAB staff for help! The ``questions" channel on Slack is also a great forum!
	
	\section{Git}
	
	Git is a version control system designed to tracking changes made to a file. Git and GitHub are extremely useful tools for writing and sharing code with your peers. Git is generally a topic that we cover second semester, but we have placed it at the start of this guide because it is a prerequisite for the next section.
	
	Download git (\hyperref{https://git-scm.com/downloads}{}{}{link}) and install onto your computer. To check that the installation was successful, open up the terminal (a program called ``terminal" on Mac and Linux and ``Command Prompt" on Windows) and enter \verb|git --version|. Your installation was successful if you do not see the message \verb|command not found|.
	
	\section{Bash}
	
	When people talk about the ``terminal" or ``command-line," they're referring to the shell: a program that serves as a text-based interface with the kernel. We'll discuss what this means in lecture! 
	
	There are many types of shells, notably Bash (Unix), DOS (Windows) and zsh (Mac). Bash, the \textbf{B}ourne-\textbf{A}gain \textbf{SH}ell\footnote{Stephen Bourne was the author of the original Unix shell, on which Bash is based. Very punny!}, is the most popular shell\footnote{Or at least, most commonly encountered.}, and the one we'll be using. Installing Bash varies by OS.
	
	\subsection{Windows}
	The default shell on Windows is DOS.  In the previous section, when you opened the ``Command Prompt" program, you were using DOS. However, we want to use Bash!
	
	Luckily, when installing git, you also installed Git Bash. You can find Git Bash in your applications folder. Git Bash is exactly what is sounds like: it is Bash shell that comes with git. In the future, when we ask us to open terminal, we mean this Bash terminal rather than the DOS Command Prompt.
	
	\subsection{Mac and Linux}
	Bash is the default shell on most Mac and Linux-based operating systems. To check if you're using Bash, open the terminal and type \verb|echo $0|. This command should return \verb|bash|. 
	
	\section{Anaconda}
	
	TODO
\end{document}