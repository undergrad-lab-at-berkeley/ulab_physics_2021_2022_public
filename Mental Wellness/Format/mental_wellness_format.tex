\documentclass[addpoints,12pt]{exam}
\usepackage[utf8]{inputenc}
\usepackage[english]{babel}
\usepackage[letterpaper, portrait, margin=1in]{geometry}
\usepackage{amsmath}
\numberwithin{equation}{section}
\usepackage{amssymb}
\usepackage{graphicx}
\usepackage{parskip}
\usepackage{xcolor}
\usepackage{physics}
\usepackage{empheq}
\usepackage{cancel}
\usepackage{hyperref}
\hypersetup{colorlinks = true, urlcolor = blue, linkcolor = red, citecolor = red}
\usepackage{enumerate}
\usepackage{tikz}
\usepackage{float}
\usepackage{tcolorbox}
\usepackage{booktabs}
\usepackage[bottom]{footmisc}

\usepackage{xcolor}
\firstpagefooter{\color{lightgray} \today}{}{Page \thepage}
\runningfooter{\color{lightgray} \today}{}{Page \thepage}

\begin{document}
	
	\begin{center}
		\textbf{\Large{ULAB Physics \& Astronomy\\Mental Wellness Workshop: Format}}
	\end{center}

	\section{Overview}
	
	The goal of the mental wellness workshop is to facilitate discussions within ULAB on issues of mental wellness. The format of the workshop is intended to be Socratic rather than lecturing or didactic. Our intention is for discussions to center on personal experience and ways to improve students' physics/astronomy experiences through their actions. 
	
	Somewhat inspired by the journal club format, each workshop is centered around an article on mental wellness. Thus, the article serves to set the topic or theme of the day.
	
	\section{The Article}
	
	The chosen article will be from a reputable source (e.g. peer-reviewed journal, or trusted news organization). The subject of the article does not have to directly relate to mental wellness, but should have some clear connection to mental wellness---particularly issues related to the physics/astronomy student experience.
	
	The article should not be overly technical. A deep understanding of article's field (e.g. psychology) should not be required.
	
	Some examples of articles and themes:
	\begin{enumerate}
		\item \textit{Very Happy People}\cite{diener2002very}: a study of what factors contribute to people's happiness over time to facilitate a discussion on social relationships.
		\item \textit{Effects of inescapable shock upon subsequent escape and avoidance responding}\cite{overmier1967effects}: a study on learned helplessness to facilitate discussion on that topic.
		\item The Challenger Disaster (topic): a review of the Challenger Disaster to facilitate discussion on group think. 
	\end{enumerate}
	
	\section{The Discussion}
	
	Each article should provoke discussion. Specifically, after reviewing the article, mentees will be encouraged to discuss/reflect on:
	\begin{enumerate}
		\item What conclusions can you draw from the article?
		\item How does the article relate to your personal experiences?
		\item Does the article confirm or refute any preconceptions you have on the topic?
		\item Does the article affect how you reflect on a past experience or circumstance?
		\item Does the article inspire you to change your mindset or behavior in the future?
	\end{enumerate}
	
	\section{The Workshop Details}
	
	The workshop leaders are vital in organizing the discussion. The workshop will begin with the leaders summarizing the article in 5-10 mins. Student will not read the article during the workshop (though we can encourage students to read the article beforehand). 
	
	The remaining time (approx. 40 mins.) will be spent discussing prompts prepared by the organizers. Students will discuss in small groups and return periodically as a whole to share.
	
	For example, a \href{https://docs.google.com/presentation/d/1po6mHtTGYhum0Xzs9pDQst9OSHkqu6Ol0Sbehd4gqHk/edit?usp=sharing}{workshop on social relationships}\footnote{A topic that was particularly relevant at the time of the workshop during the COVID-19 pandemic.} could have the format:
	\begin{enumerate}
		\item (10 min) Organizers summarize article
		\item (10 min) What did you get out of the article? Are there any interesting/surprising results?
		\item (10 min) Group discussion then share: According ot the article, what correlates to high happiness? Does happiness require the absence of negative experiences?
		\item (10 min) Group discussion then share: What is a social relationship and how does it relate to happiness?
		\item (10 min) Group discussion then share: How can I use this knowledge. What are some actions I can take to develop social relationships?
	\end{enumerate}
	
	Ideally, organizers should be willing to share their own experiences/thoughts on the topic to jump-start the conversation. Additionally, a system in which attendees can anonymously share their thoughts might be beneficial\footnote{I.e. anonymous Google form where students can submit responses to be discussed.}.	
	
	\bibliographystyle{ieeetr}
	\bibliography{refs.bib}
	\nocite{*}


\end{document}