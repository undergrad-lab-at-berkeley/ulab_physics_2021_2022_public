\documentclass[addpoints,12pt]{exam}
\usepackage[utf8]{inputenc}
\usepackage[english]{babel}
\usepackage[letterpaper, portrait, margin=1in]{geometry}
\usepackage{amsmath}
\numberwithin{equation}{section}
\usepackage{amssymb}
\usepackage{graphicx}
\usepackage{parskip}
\usepackage{xcolor}
\usepackage{physics}
\usepackage{empheq}
\usepackage{cancel}
\usepackage{hyperref}
\hypersetup{colorlinks = true, urlcolor = blue, linkcolor = red, citecolor = red}
\usepackage{enumerate}
\usepackage{tikz}
\usepackage{float}
\usepackage{tcolorbox}
\usepackage{booktabs}
\usepackage[bottom]{footmisc}
\usepackage{booktabs}
\renewcommand{\arraystretch}{1.5}


\usepackage{xcolor}
\firstpagefooter{\color{lightgray} \today}{}{Page \thepage}
\runningfooter{\color{lightgray} \today}{}{Page \thepage}

\begin{document}
	
	\begin{center}
		\textbf{\Large{ULAB Physics \& Astronomy\\Mental Wellness Workshop: Format}}
	\end{center}

	\section{Overview}
	
	The goal of ULAB is to provide students with a well-rounded preparation for pursing research. Beyond technical skills, students' mental wellness is an important part of research and undergraduate success. Our approach is two-fold:
	
	First, at the start of each academic year, organizations such as Path to Care are invited to give students' a qualified-overview of serious issues such as sexual harassment, power dynamics in a research environment, general mental health, etc.
	
	While these issues are relevant, they tend to overlook the more mundane aspects of mental wellness and specifically as they relate to physics and astronomy students. For example, 
	\begin{enumerate}
		\item \textbf{Sense of belonging}: \textit{how do I interact with my physics peers? My professor expects me to have prior knowledge on tensor indices\footnote{Honestly, what even is a tensor?}, and everyone except me understands it. I thought I liked physics, but I'm doing so poorly, maybe physics isn't right for me?}
		\item \textbf{Feeling inadequate}: \textit{my peers are getting better grades, taking harder classes, have had so much research opportunities...}
		\item \textbf{Overwhelmed by the coursework:} \textit{I don't know how to even start the homework and the GSI hasn't been very helpful. People ask really advanced questions in class and I'm afraid to ask my ``stupid" question. I'm doing fine in my classes, but I just have too much work.}
	\end{enumerate}
	
	We hope to promote mental wellness in this regard by fostering physics/astronomy-student  lead conversations on such issues. The format of these additional workshops is intended to be Socratic rather than didactic. Through discussions that center on personal experience, we wish to show students that they are not alone in their struggles and to brainstorm ways of improving students' physics/astronomy experiences through their actions. \textbf{Workshops are hosted by ULAB/physics/astronomy peers.}
	
	\section{Format}
	
	Somewhat inspired by the journal club format, each workshop is centered around an article related to a topic in mental wellness. The article\footnote{The article should be from a reputable source and not be overly technical. A deep understanding of article's field (e.g. psychology) should not be required.} serves to set the topic or theme of the day.
	
	Some examples of past articles and themes:
	\begin{enumerate}
		\item \textit{Very Happy People}\cite{diener2002very}: a study of what factors contribute to people's happiness over time to facilitate a discussion on social relationships within the physics/astronomy community.
		\item \textit{Effects of inescapable shock upon subsequent escape and avoidance responding}\cite{overmier1967effects}: a study on learned helplessness to facilitate discussion on that topic.
		\item The Challenger Disaster (topic): a review of the Challenger Disaster to facilitate discussion on group think. 
	\end{enumerate}

	\subsection{Timeline}
	
	\begin{table}[h!]
	\begin{tabular}{p{4cm}|p{3cm}|p{9cm}}
		& \textbf{Time (mins.) }& \textbf{Notes        }                                                                                                                                                                         \\
		\hline
		Introduction      & 5            & Setting up a safe space.                                                                                                                                                              \\
		\hline
		Article Summary   & 5            & General info: researchers’ affiliations, key definitions, questions/hypothesis they hoped to answer. Methodology: data collection, analysis, and findings.                            \\
		\hline
		Why Should I Care? & 5            & Societal trends or key example. Good place to add polls and ask trick questions                                                                                                       \\
		\hline
		Discussion        & 20           & Relevance of research to each mentee or to any physics undergrad? How to take action in physics/undergrad settings? Lots of mentee discussion, guided questions, actionable takeaways \\
		\hline
		Q\&A              & 5            & Mentees are encouraged to share or ask questions at any point in time. This is another opportunity at the end.                                                                       
		\end{tabular}
	\end{table}

	\subsection{Section: Introduction}
	
	In the past, the following points were reviewed at the start of the workshop:
	\begin{enumerate}
		\item\textbf{ Equalize the space:}
		\begin{enumerate}
			\item All knowledge and opinions are equally valid
			\item Share stories and experiences, not names/gossip
			\item Give space before taking space
		\end{enumerate}
		\item \textbf{Check your assumptions}
		\begin{enumerate}
			\item No judgements (including self-judgements)
			\item Treat everyone as an individual and not a representative of a specific group
			\item Believe in our common best intentions
		\end{enumerate}
		\item \textbf{The right to be human}
		\begin{enumerate}
			\item Avoid blaming people for misinformation taught to them
			\item Acknowledge emotions
			\item Practice forgiveness
		\end{enumerate}
		\item \textbf{Practice consensual dialogue}
		\begin{enumerate}
			\item Active listening
		\end{enumerate}
	\end{enumerate}
	
	The peer workshop organizers also acknowledged that they were simply invested undergraduate students and not trained professionals. What they discuss at a workshop is based on their research on the subject and their and our own experiences as students in the undergraduate communities at Berkeley.
		
	\subsection{Section: Article Summary}

	There is not enough time in the workshop for students' to read the article. Optionally, students can read the article ahead of time, but a summary will be provided during the workshop.

	\subsection{Section: Why Should I Care?}
	
	Pretty self-explanatory. Article summaries tend to be more technical. But how does the article relate to the physics/astronomy student experience? \textbf{You can be creative!} For example, \textit{you should care about the Challenger disaster because just as the engineers were afraid to speak up about failure source, many students are afraid to speak up during class.}
	
	\subsection{Section: Discussion}
	
	This is the ``meat" of the workshop. After reviewing the article, the peer organizers will guide mentees to discuss/reflect on:
	\begin{enumerate}
		\item What conclusions can you draw from the article?
		\item How does the article relate to your personal experiences?
		\item Does the article confirm or refute any preconceptions you have on the topic?
		\item Does the article affect how you reflect on a past experience or circumstance?
		\item Does the article inspire you to change your mindset or behavior in the future?
	\end{enumerate}

	This can be achieved with a number of formats: (1) whole group discussion, (2) one question at a time, breakout into small groups to discuss, then report to entire group, etc.
	
	Ideally, organizers should be willing to share their own experiences/thoughts on the topic to jump-start the conversation. Additionally, a system in which attendees can anonymously share their thoughts might be beneficial\footnote{I.e. anonymous Google form where students can submit responses to be discussed.}.	
	
	\bibliographystyle{ieeetr}
	\bibliography{refs.bib}
	\nocite{*}


\end{document}