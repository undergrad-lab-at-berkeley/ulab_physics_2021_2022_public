\documentclass[addpoints,12pt]{exam}
\usepackage[utf8]{inputenc}
\usepackage[english]{babel}
\usepackage[letterpaper, portrait, margin=1in]{geometry}
\usepackage{amsmath}
\numberwithin{equation}{section}
\usepackage{amssymb}
\usepackage{graphicx}
\usepackage{parskip}
\usepackage{xcolor}
\usepackage{physics}
\usepackage{empheq}
\usepackage{cancel}
\usepackage{hyperref}
\hypersetup{colorlinks = true, urlcolor = blue, linkcolor = red, citecolor = red}
\usepackage{enumerate}
\usepackage{tikz}
\usepackage{float}
\usepackage{tcolorbox}
\usepackage{booktabs}
\usepackage[bottom]{footmisc}

\usepackage{xcolor}

\firstpagefooter{\color{lightgray} \today}{}{Page \thepage}
\runningfooter{\color{lightgray} \today}{}{Page \thepage}

\begin{document}
	
	\vspace*{-3cm}{\footnotesize\hfill Copyright \copyright\ 2021, Undergraduate Lab at Berkeley}
	\vspace{0.5cm}
	
	\begin{center}
		\textbf{\Large{ULAB Physics \& Astronomy\\Python Module 1}}
	\end{center}
	
	\begin{questions}
		
		\question[0] Follow instructions on the installation guide to install Git (section 2), Bash (section 3), and Anaconda (section 4). You will need these for this module and for next week!
		
		\question[10] Bash commands follow the syntax \\\verb|[command] [flags/options] [other arguments]|\\Some commands have fixed behavior and do not require any additional parameters. 
		
		However, you will often need to specify arguments. For example, \verb|cd |$\sim$\verb|/Desktop| runs the command \verb|cd| (change directory) with the argument \verb|cd |$\sim$\verb|/Desktop|.
		
		In addition to arguments, we use flags to indicate that a command should behave differently. Flags are dashes followed by flag name. Each command has it's own set of flags and options. For example, the default (no flags) behavior of \verb|ls| is to list the files in your current directory. When we include the list-all flag  \verb|ls -a| the program will list all files, including those that are normally hidden.
		
		
		For this exercise, describe the function of each of the following Bash commands. Explain what arguments you need to provide (if any). Hint: \hyperref{https://ss64.com/bash/}{}{}{https://ss64.com/bash/}
		
		\begin{parts}
			\part \verb|man|
			\part \verb|pwd| (hint: if you're confused about a directory is, read question 2)
			\part \verb|cd|
			\part \verb|ls -l|
			\part \verb|mkdir|
			\part \verb|rm| and \verb|rm -r| (hint: one is for files, one is for directories)
			\part \verb|mv|
			\part \verb|cat|
		\end{parts}
		
		Test out of these commands on the terminal! (Be careful with \verb|rm|, there is no recycle bin in Bash)
		
		\newpage
		
		\question[10] In Bash, we call folders ``directories." Directories can contain other directories or files. The root directory is the directory on your computer that contains all other directories and files. Often times, we will find the Applications, Users, etc. folders in the root directory.
		
		We can specify the path (location) of a file as an absolute path or a relative path. The absolute path is the location of a file from the root directory. The relative path is the location of a file from the current working directory (which you can determine with the command \verb|pwd|).
		
		Bash defines special symbols for particularly useful directories:
		
		``\verb|/|" root directory
		
		``\verb|.|" current working directory
		
		``\verb|..|" one directory above
		
		\begin{figure}[H]
			\centering
			\includegraphics[width=6cm] {dir.png}
		\end{figure}
		
		For example, \verb|foo.txt| has absolute path \verb|/User/ulab/foo.txt|. If our current directory is \verb|/User/Shared|, then the relative path is \verb|../ulab/foo.txt| (make sure you understand why this is the case!)
		
		For the following exercises, the current directory is \verb|/User/Shared|.
		
		\begin{parts}
			\part What is the absolute path to the directory \verb|baz.txt|?
			
			\part What is the absolute and relative path to \verb|bar.pdf|?
			
			\part What is the relative path to \verb|/System|?
			
			\part The \verb|cd| command accepts both absolute and relative paths. What is (one) command to navigate to the ulab folder?
			
			\part Is \verb|/User/ulab/bar.pdf| a directory?
			
		\end{parts}

		\vspace{.5in}
	
	\end{questions}
\end{document}