\documentclass[12pt]{article}
\usepackage[utf8]{inputenc}
\usepackage[english]{babel}
\usepackage[letterpaper, portrait, margin=1in]{geometry}
\usepackage{amsmath}
\numberwithin{equation}{section}
\usepackage{amssymb}
\usepackage{graphicx}
\usepackage{parskip}
\usepackage{xcolor}
\usepackage{physics}
\usepackage{empheq}
\usepackage{cancel}
\usepackage{hyperref}
\hypersetup{colorlinks = true, urlcolor = blue, linkcolor = red, citecolor = red}
\usepackage{enumerate}
\usepackage{tikz}
\usepackage{float}
\usepackage{tcolorbox}
\usepackage{booktabs}
\usepackage[bottom]{footmisc}

% Default fixed font does not support bold face
\DeclareFixedFont{\ttb}{T1}{txtt}{bx}{n}{12} % for bold
\DeclareFixedFont{\ttm}{T1}{txtt}{m}{n}{12}  % for normal

% Custom colors
\usepackage{color}
\definecolor{deepblue}{rgb}{0,0,0.5}
\definecolor{deepred}{rgb}{0.6,0,0}
\definecolor{deepgreen}{rgb}{0,0.5,0}

\usepackage{listings}

% Python style for highlighting
\newcommand\pythonstyle{\lstset{
language=Python,
basicstyle=\ttm,
morekeywords={self},              % Add keywords here
commentstyle=\color{gray},
keywordstyle=\ttb\color{deepblue},
emph={MyClass,__init__},          % Custom highlighting
emphstyle=\ttb\color{deepred},    % Custom highlighting style
stringstyle=\color{deepgreen},                        % Any extra options here
showstringspaces=false
}}


% Python environment
\lstnewenvironment{python}[1][]
{
\pythonstyle
\lstset{#1}
}
{}

% Python for external files
\newcommand\pythonexternal[2][]{{
\pythonstyle
\lstinputlisting[#1]{#2}}}

% Python for inline
\newcommand\pythoninline[1]{{\pythonstyle\lstinline!#1!}}

\usepackage{xcolor}
\usepackage{fancyhdr}
\pagestyle{fancy}
\fancyhf{}
\fancyfoot[C]{\color{lightgray} Python Lecture V Notes---Yi J Zhu}
\fancyfoot[L]{\color{lightgray} \today}
\fancyfoot[R]{Page \thepage}
\renewcommand{\headrulewidth}{0pt}
\renewcommand{\footrulewidth}{0pt}
\begin{document}

\section{Review}

\textbf{Last time:}
\begin{itemize}
    \item Python miscellany
    \item Questions?
    \item Today: Numpy
\end{itemize}

\section{Numpy Basics}
\begin{itemize}
    \item NumPy (Numerical Python) is a python package for working with numerical data
    \item numpy is the standard for data analysis and is used in MANY other scientific packages---it's very important that you're familiar with Python
    \item Numpy should come with your Anaconda distribution.
    \item If you don't have numpy, install on the command line: \verb|pip install numpy|
    \item To import numpy
    \begin{python}
    import numpy as np
    \end{python}
\end{itemize}

\section{ndarray}
\begin{itemize}
    \item An n-dimensional array of items of the SAME type.
    \item ndarrays are faster and take up less memory than Python lists
    \item It is MUCH easier to perform calculations using ndarrays (we'll discuss this more!)
    \item numpy arrays have a shape given by a tuple. \verb|(rows, cols)| for 2d arrays.
\end{itemize}

\textbf{ndarray example}:
\begin{python}
import numpy as np
a = array([[ 0,  1,  2,  3,  4],
           [ 5,  6,  7,  8,  9],
           [10, 11, 12, 13, 14]])

>>> a.shape
(3,5)
>>> a.ndim
2
>>> a.dtype.name
'int64'
>>> a.size
15
\end{python}

\textbf{Creating arrays:}
\begin{itemize}
    \item We can create arrays from an existing list \pythoninline{a = np.array([[2,3,4], [5,6,7])}
    \item Ones and zeros:
    \begin{python}
    >>> np.zeros((3, 4))
    array([[0., 0., 0., 0.],
           [0., 0., 0., 0.],
           [0., 0., 0., 0.]])
           
    >>> # dtype can also be specified
    >>> # This is a 3d array! (we'll talk about axes in a bit!)
    >>> np.ones((2,3,4), dtype=np.int16) 
    array([[[1, 1, 1, 1],
            [1, 1, 1, 1],
            [1, 1, 1, 1]],
    
           [[1, 1, 1, 1],
            [1, 1, 1, 1],
            [1, 1, 1, 1]]], dtype=int16)
    \end{python}
    \item Create a seqeunce:
    \begin{python}
    # evenly spaced range arange([start], stop, [step])
    >>> np.arange( 10, 30, 5 )
    array([10, 15, 20, 25])
    
    # specify a number of elements across a range 
    # linspace(start, stop, num)
    >>> np.linspace(0, 1, 5)
    array([0.  , 0.25, 0.5 , 0.75, 1.  ])
    \end{python}
\end{itemize}

\textbf{numpy axes}:
    \begin{itemize}
        \item Often a very confusing topic!
        \begin{figure}[H]
	    \centering
	    \includegraphics[width=6.5cm] {axes}
        \end{figure}
        
        \item Each dimension corresponds to an axis
        \item The axis convention is \textbf{OUT TO IN}. For example: the 3d array in the figure above has 3 degrees of freedom:
        \begin{itemize}
            \item \textbf{axis-0}: which one of the \textbf{two} 2d arrays? (\verb|a[0]| or \verb|a[1]|)
            \item \textbf{axis-1}: given one of the 2d arrays, which of the \textbf{three} 1d arrays (\verb|a[i,0]| or \verb|a[i,1]| or \verb|a[i,2]|)
            \item \textbf{axis-2:} given one of the 1d arrays, which of the \textbf{four} elements? (\verb|a[i,j,0]| or \verb|a[i,j,1]| or \verb|a[i,j,2]| or \verb|a[i,j,3]|)
        \end{itemize}
        \item For 2d array: \verb|(row, col)|. For 3d array: \verb|(matrix, row, col)|.
        \item This is why the array has shape \verb|(2,3,4)|
    \end{itemize}
    
\textbf{Practice with array indexing: }
\begin{python}
>>> a = np.arange(24).reshape((2,3,4))
>>> a
array([[[ 0,  1,  2,  3],
        [ 4,  5,  6,  7],
        [ 8,  9, 10, 11]],

       [[12, 13, 14, 15],
        [16, 17, 18, 19],
        [20, 21, 22, 23]]])

# What is a[1]?
# What is a[1,2]?
# What is a[1,2,3]

# Notice: array slicing still works!
# Notice: a[1] = a[1,:,:] = a[1:...]

# REMEMBER: read left to right, out to in
# What is a[0,0,2:]?
# What is a[0,:,2:]?
# What is a[:,:2,2:]?
\end{python}

\textbf{Manipulating arrays} (you can read up on your own):
\begin{itemize}
    \item \verb|np.reshape|
    \item \verb|np.concatenate|
    \item \verb|np.stack|, \verb|np.vstack, np.hstack, np.vsplit, np.hsplit|, \verb|np.column_stack|
\end{itemize}

\section{Vectorized Operations}
\begin{itemize}
    \item Often times, we want to perform element-wise operations on an array. (For example, we want to multiply all elements by some factor). This is called a VECTORIZED operation.
    \item How would we do this with lists?
    \begin{python}
    a = [[1,2,3], [4,5,6]]
    # what happens when we do 3*a? (it ends badly)
    
    for row in range(len(a)):
        for col in range(len(a[0])):
            a[i][j] = 3 * a[i][j]
    \end{python}
    This is SLOW and ANNOYING to write. It's also hard to quickly understand what the code is doing!
    \item In numpy, mathematical operations are vectorized!
    \begin{python}
    a = np.array(a)
    a = 3 * a
    
    >>> a
    array([[ 3,  6,  9],
           [12, 15, 18]])
    \end{python}
    Numpy's vectorized operations are implemented in C, so they are FAST! With numpy, it is easier to perform element-wise operations and it's easier to read!
\end{itemize}
\begin{figure}[H]
	\centering
	\includegraphics[width=16.5cm] {vector}
\end{figure}

\textbf{Example:}
\begin{python}
>>> x = np.arange(5)
>>> x
array([0, 1, 2, 3, 4])

>>> y = 5*x + 1
>>> y
array([ 1,  6, 11, 16, 21])

# operations between two numpy arrays are STILL element-wise
>>> x + y
array([ 1,  7, 13, 19, 25])
>>> x * y
array([ 0,  6, 22, 48, 84])

>>> np.max(y)
21
>>> np.mean(y)
11.0
\end{python}
\textbf{We can specify the axis of an operation:}
\begin{python}
>>> a = np.arange(24).reshape((2,3,4))
>>> a
array([[[ 0,  1,  2,  3],
        [ 4,  5,  6,  7],
        [ 8,  9, 10, 11]],

       [[12, 13, 14, 15],
        [16, 17, 18, 19],
        [20, 21, 22, 23]]])

>>> np.sum(a)
276
>>> # sum ALONG the 0 axis
>>> np.sum(a, axis=0)
array([[12, 14, 16, 18],
       [20, 22, 24, 26],
       [28, 30, 32, 34]])
\end{python}

\textbf{Numpy also supports vectorized logical operations: }
\begin{python}
>>> a = np.arange(24).reshape((2,3,4))
>>> a%2 == 0
array([[[ True, False,  True, False],
        [ True, False,  True, False],
        [ True, False,  True, False]],

       [[ True, False,  True, False],
        [ True, False,  True, False],
        [ True, False,  True, False]]])
>>> np.where(a%2 == 0)
(array([0, 0, 0, 0, 0, 0, 1, 1, 1, 1, 1, 1]), 
 array([0, 0, 1, 1, 2, 2, 0, 0, 1, 1, 2, 2]), 
 array([0, 2, 0, 2, 0, 2, 0, 2, 0, 2, 0, 2]))
>>> a[np.where(a%2 == 0)]
array([ 0,  2,  4,  6,  8, 10, 12, 14, 16, 18, 20, 22])
\end{python}

\section{Additional}
Numpy has much more functionality that we won't have the chance to cover (to name a few):
\begin{itemize}
    \item Random
    \item Linear algebra (linalg)
    \item Fast Fourier Transfor (fft)
\end{itemize}

\textbf{Read more about Numpy:}
\begin{itemize}
    \item \url{https://numpy.org/doc/stable/contents.html}
    \item \url{https://www.pythonlikeyoumeanit.com/module_3.html}
\end{itemize}   

\section{Next Time}
\begin{itemize}
    \item Pandas
    \item Plotting in python with Matplotlib
\end{itemize}

\appendix
\section{(Optional) Array Broadcasting}
When two arrays have the same shape, we can easily understand what an element-wise operation does---it applies some operation between corresponding elements of the arrays!

For example (both arrays have \verb|shape=(2,3)|),
\begin{equation*}
    \begin{pmatrix}
        1 & 2 & 3\\
        4 & 5 & 6\\
    \end{pmatrix}
    +
    \begin{pmatrix}
        1 & 1 & 1\\
        2 & 2 & 2\\
    \end{pmatrix}
    \xrightarrow{element-wise }
    \begin{pmatrix}
        2 & 3 & 4\\
        6 & 7 & 8\\
    \end{pmatrix}
\end{equation*}

However, we can use BROADCASTING to perform operations on arrays of UNEQUAL shapes IF they are broadcastable. For example, adding a \verb|(3,2)| array to a \verb|(2,)| array:

\begin{equation*}
    \begin{pmatrix}
        1 & 2\\
        3 & 4\\
        5 & 6\\
    \end{pmatrix}
    +
    \begin{pmatrix}
        1 & 2\\
    \end{pmatrix}
    \xrightarrow{broadcast }
    \begin{pmatrix}
        1 & 2\\
        3 & 4\\
        5 & 6\\
    \end{pmatrix}
    +
    \begin{pmatrix}
        1 & 2\\
        1 & 2\\
        1 & 2\\
    \end{pmatrix}
    \xrightarrow{element-wise }
	\begin{pmatrix}
		2 & 4\\
		4 & 6\\
		6 & 8\\
	\end{pmatrix}
\end{equation*}

\textbf{Broadcasting steps} (using example above):
\begin{enumerate}
    \item Determine the shape of both arrays: \verb|(3,2)| and \verb|(2,)|
    \item If the shapes are not the same size, \textbf{prepend} the smaller shape with 1's until both arrays have the same number of dimensions. In this case, the first shape has dimension 2, so we must extend the second shape: \verb|(2,)| $\rightarrow$ \verb|(1,2)|
    \item Align the array dimensions:
    \begin{align*}
        &\text{(3, 2)}\\
        &\text{(1, 2)}
    \end{align*}
    Each of the respective dimensions must be (a) equal or (b) unequal but one dimension is 1. \textbf{Otherwise, the arrays can not be broadcast.} In this case, axis-0 (3 and 1) are not equal but contains one 1, and axis-1 (2 and 2) are equal.
    \item For each of the dimensions, if the respective dimensions are equal, the operation is applied element wise. If the respective dimensions are not equal, but one dimension is 1, then the dimension of size 1 is broadcast to the larger dimension.
\end{enumerate}

   This is fully generalized. However, broadcasting can get nasty with high dimensions. As always, if you're getting lost in the complexity of your data structure, you should probably reconsider it's format. 
   
  
  \section{(Optional) Python Speedups Using C}
  
  Earlier, we mentioned that Numpy is fast because it is implemented in C. What does this mean? After all, Numpy is a \textit{Python} package. 
  
  In short, the core functionality of Numpy (ndarrays and vectorized operations) is written in the programming language C with C syntax, C variables, etc. However, Numpy provides Python wrapper functions that allows the user to call Python functions (e.g. np.ndarray()) and perform operations in what is really C behind the scenes.
  
  Why are we going to all this trouble? \textit{Speed.} Python has considerably more overhead than C. This means that for the same basic operation, Python needs to perform more work and use more memory than C. 
  
  For example, Python is dynamically typed: a variable can hold any data type. While this is convenient for us, the programmer, it takes additional memory and operations behind-the-scenes to implement. On the other hand, C is strongly typed: you must define beforehand the variable type. Thus, the program will only allocate as much memory as the type requires and there is no run-time ambiguity about the data type a variable contains. This is reflected in the fact that  Numpy arrays can only store a single type of data.
  
  While these optimizations may seem minor, they add up. For example, let's compare a function that computes the Fibonacci sequence written in both C and Python. First, we implement \verb|fib()| in C,
  
  \begin{python}
	# fib.c
	int fib(int n) {
	    if(n<2) {
	       return n;
	    }
	    return fib(n-1) + fib(n-2)
	}
  \end{python}
	
	Making use of the \verb|ctypes| module, we define a wrapper function \verb|c_fib()| to use the C-implemented \verb|fib| function (above) in Python. 

\begin{python}
	# fib.py
	import time
	import ctypes
	
	# C implementation of fib with Python wrapper
	# gcc -shared -o fib.so -fPIC fib.c
	_libc = ctypes.CDLL('fib.so')
	
	# one way to do this
	# c_fib = _libc.fib
	# c_fib.restype = ctypes.c_int
	# c_fib.argtypes = [ctypes.c_int]
	
	def c_fib(a, b):
	    return libc.fib(c_int(a), c_int(b))
	
	
\end{python}

Let's now implement the exact same function, but in Python.

\begin{python}
	# python implementation of fib 
	def py_fib(n):
	    if n<2:
	        return n
	    return py_fib(n-1) + py_fib(n-2)
	
\end{python}
Benchmarking the run times,
\begin{python}
	start = 0
	def tik():
	    global start
	    start = time.time()
	def tok():
	    end = time.time()
	    print(f'\t{end-start:3f} sec')
	
	
	print('py_fib(36): ')
	tik()
	print(f'\t{py_fib(36)}')
	tok()
	
	print('c_fib(36): ')
	tik()
	print(f'\t{c_fib(36)}')
	tok()
	
	>>> python fib.py
	py_fib(36): 
	    14930352
	    time: 5.697 sec
	c_fib(36): 
	    14930352
	    time: 0.110 sec
\end{python}
  
\verb|c_fib| is considerably faster than \verb|py_fib|---this is why packages such as Numpy and Scipy are so popular. We can take advantage of the speedups of using C while retaining the convenience of writing code in Python!
	\end{document}
