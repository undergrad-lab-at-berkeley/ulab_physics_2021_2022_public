\documentclass[12pt]{article}
\usepackage[utf8]{inputenc}
\usepackage[english]{babel}
\usepackage[letterpaper, portrait, margin=1in]{geometry}
\usepackage{amsmath}
\numberwithin{equation}{section}
\usepackage{amssymb}
\usepackage{graphicx}
\usepackage{parskip}
\usepackage{xcolor}
\usepackage{physics}
\usepackage{empheq}
\usepackage{cancel}
\usepackage{hyperref}
\hypersetup{colorlinks = true, urlcolor = blue, linkcolor = red, citecolor = red}
\usepackage{enumerate}
\usepackage{tikz}
\usepackage{float}
\usepackage{tcolorbox}
\usepackage{booktabs}
\usepackage[bottom]{footmisc}

% Default fixed font does not support bold face
\DeclareFixedFont{\ttb}{T1}{txtt}{bx}{n}{12} % for bold
\DeclareFixedFont{\ttm}{T1}{txtt}{m}{n}{12}  % for normal

% Custom colors
\usepackage{color}
\definecolor{deepblue}{rgb}{0,0,0.5}
\definecolor{deepred}{rgb}{0.6,0,0}
\definecolor{deepgreen}{rgb}{0,0.5,0}

\usepackage{listings}

% Python style for highlighting
\newcommand\pythonstyle{\lstset{
language=Python,
basicstyle=\ttm,
morekeywords={self},              % Add keywords here
commentstyle=\color{gray},
keywordstyle=\ttb\color{deepblue},
emph={MyClass,__init__},          % Custom highlighting
emphstyle=\ttb\color{deepred},    % Custom highlighting style
stringstyle=\color{deepgreen},                        % Any extra options here
showstringspaces=false
}}


% Python environment
\lstnewenvironment{python}[1][]
{
\pythonstyle
\lstset{#1}
}
{}

% Python for external files
\newcommand\pythonexternal[2][]{{
\pythonstyle
\lstinputlisting[#1]{#2}}}

% Python for inline
\newcommand\pythoninline[1]{{\pythonstyle\lstinline!#1!}}

\usepackage{xcolor}
\usepackage{fancyhdr}
\pagestyle{fancy}
\fancyhf{}
\fancyfoot[C]{\color{lightgray} Python Lecture IV Notes}
\fancyfoot[L]{\color{lightgray} \today}
\fancyfoot[R]{Page \thepage}
\renewcommand{\headrulewidth}{0pt}
\renewcommand{\footrulewidth}{0pt}
\begin{document}

\section{Review}

\textbf{Last time:}
\begin{itemize}
    \item Conditionals, loops, functions, errors
    \item Questions?
\end{itemize}

\textbf{Today:} we will discuss a miscellany of topics. This will round off the basic intro to Python. After this lecture we'll be discussing very important packages for scientific computing: numpy, scipy, matplotlib, pandas

\section{List Comprehensions}
\begin{itemize}
    \item Shorthand for creating a NEW list BASED on an existing list.
    \item Iterate through list and perform an operation given an optional condition
    \item Syntax is similar to a for loop
    \verb|[EXPRESSION for x in ITERABLE if CONDITION]|
    \item They sometimes make your life easier. Downside: code is harder to read
\end{itemize}

\begin{python}
lst = [3, 1, 4, 1]
for i in range(len(lst)):
    lst[i] = lst[i]*2

# with list comprehension
double = [x*2 for x in lst]

# keep only values greater than 
new_lst = [x for x in lst if x>1]

# create a list of prime numbers that are between 1 and 1000
primes = [x for x in range(1, 1001) if is_prime(x)]
\end{python}

\section{2D Arrays}
A list is an ordered collection of objects. A list is an object. You can have a list of lists. A list of lists is a two-dimensional list/array (list and array are often interchangeable...)

\begin{python}
sigma_x = [[0, 1], [1, 0]]

sigma_x = [ [0, 1], 
            [1, 0] ]
\end{python}

\begin{itemize}
    \item What is \verb|sigma_x[0]|?
    \item What is \verb|sigma_x[0][0]|?
    \item Syntax: \verb|array[row][col]|
    \item Can have higher dimensional arrays. But if you NEED to use a higher dimensional array to store data, you should probably define a class (object orientated programming) or find a better way to format your data.
\end{itemize}

\section{Tuples}
\textbf{Lists:} an ordered collection that allows duplicates

\textbf{Tuples:} an ordered collection with duplicates BUT is immutable (can not be changed after creation).
\begin{itemize}
    \item Tuples are denotes by parenthesis \verb|()|
    \item Single tuple denoted by \verb|(element,)|. Comma indicates tuple.
    \item Tuples are faster than lists (no dynamic memory allocation)
    \begin{python}
    tup = (3, 1, 4)
    
    >>> tup[0]
    3
    >>> tup[0] = 5
    Traceback (most recent call last):
    File "<stdin>", line 1, in <module>
    TypeError: 'tuple' object does not support item assignment
    
    tup = ('1-tuple',)
    
    >>> len(tup)
    1
    \end{python}
\end{itemize}

\section{Dictionaries}
\textbf{Dictionary: }an unordered, mutable collection of key:value pairs. 
\begin{itemize}
    \item Keys must be unique and immutable (string, tuple, number)
    \item Why is this data type called a dictionary?
\end{itemize}
\begin{python}
dict = {
    'name' = 'ulab',
    'age' = 5
}
\end{python}
\begin{python}
# view dictionary contents
print(dict)

# method returns array of keys
keys = dict.keys()

# checking if a key exists
contains = 'age' in dict

# accessing key value
age = dict['age']

# if KEY does not already exists in the dictionary
# a new key:value pair is created
dict['email'] = 'physics.ulab@gmail.com'

# if KEY already EXISTS in the dictionary
# the key value is updated
dict['email'] = 'physics@ulab.berkeley.edu'

# pop removes the key:value pair and returns the value
dict.pop('email')

# looping through dictionaries
for key in dict:
    print(key)

\end{python}

\section{Lambda Functions}

\textbf{Setup: functions can have functions as arguments}
\begin{python}
def map(f, iter):
    return [f(x) for x in iter]
    
def triple(x):
    return e*x

lst = [3, 1, 4]
lst_new = map(triple, lst)

>>> lst_new
[9, 3, 12]
\end{python}

\textbf{Lambda functions:}
\begin{itemize}
    \item Sometimes we want to define small ANONYMOUS functions
    \item Anonymous: function does not have a name
    \item Lambda functions are most often used when passing in a function as an argument
    \item Lambda functions can take any number of arguments, but must return a single expression (no conditionals loops).
    \item Syntax:
    \begin{python}
    lambda [args] : [expression]
    \end{python}
    \item Example:
    \begin{python}
    lst_new = map(lambda x: 3*x, lst)
    
    # you can assign a variable the value of a lambda function
    triple = lambda x: 3*x
    add = lambda x, y: x+y
    
    >>> add(1, 2)
    3
    \end{python}
\end{itemize}

\section{Importing}

\textbf{Modules:} a Python file containing definitions for variables, functions, and classes (we wont' discuss object oriented programming):
\begin{itemize}
	\item A module IS a .py file. There is nothing special about a module.
	\item A module contains code that you or someone else has written to serve a specific purpose.
	\item We IMPORT modules when we want to use their functionality
	\item Python has a ton of built-in modules: math, random, sys, etc...
	\item It is convention to place import statement at the start of your code
\end{itemize}

\textbf{Our own module example:}
\begin{python}
# file: mathtools.py

pi = 3.14
e = 2.718

# this function is recursive...but we could also use a for loop
def fact(x):
	if x==0:
		return 1
	return x*fact(x-1)

# calculates e^x using Taylor series
def exp(x):
	val = 0
	for n in range(30):
		val += (x**n) / fact(n)
	return val
\end{python}

Now, let's import!

\begin{python}
# basic import
import mathtools
>>> mathtools.pi
3.14
>>> mathtools.fact(3)
6

# import specific names
from mathtools import fact
>>> fact(3) 
6

# import all names
from mathtools import *
>>> pi
3.14
>>> e
2.718

# import and rename
from mathtools import fact as f
>>> f(3)
6

# Python has a built-in math module with exp defined
import math
>>> math.exp(3.14)
23.103866858722185
>>> mathtools.exp(3.14)
23.103866858722185
	
\end{python}

\textbf{Packages: }a directory containing a collection of modules and sub-packages.
\begin{verbatim}
/package
|   __init__.py
|   module1.py
|   module2.py
|   /sub-package1
    |   __init__.py
    |   module3.py
\end{verbatim}
\begin{itemize}
    \item A package must contain \verb|__init__.py| 
    \item Importing packages work the same way as importing modules
    \item Packages are used to share 
    \item We will use the Numpy, Matplotlib, Pandas, and Scipy packages!
\end{itemize}

\textbf{pip}: a package installer for Python. pip installs packages located on the Python Package Index (PyPI) repository. On the command line: \verb|pip install PACKAGE_NAME|

\section{Next Time}
\begin{itemize}
    \item Numpy!
\end{itemize}

\end{document}
